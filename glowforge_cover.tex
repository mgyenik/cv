\documentclass[11pt]{article}

\usepackage[T1]{fontenc}
\usepackage{lmodern}
\usepackage{marvosym}
\usepackage{graphicx}


\pagestyle{empty}

\usepackage[scale=0.775]{geometry}
\setlength{\parindent}{0pt}
\addtolength{\parskip}{6pt}

\def\firstname{Michael}
\def\familyname{Yenik}
%\def\FileAuthor{\firstname \familyname}
%\def\FileTitle{\firstname \familyname's cover letter}
%\def\FileSubject{Cover letter}
%\def\FileKeyWords{\firstname \familyname, Cover letter}
%
%\renewcommand{\ttdefault}{pcr}

%\usepackage{url}
%\urlstyle{tt}
%\ifpdf
%  \usepackage[pdftex,pdfborder=0,breaklinks,baseurl=http://,pdfpagemode=None,pdfstartview=XYZ,pdfstartpage=1]{hyperref}
%  \hypersetup{
%    pdfauthor   = \FileAuthor,%
%    pdftitle    = \FileTitle,%
%    pdfsubject  = \FileSubject,%
%    pdfkeywords = \FileKeyWords,%
%    pdfcreator  = \LaTeX,%
%    pdfproducer = \LaTeX}
%\else
%  \usepackage[dvips]{hyperref}
%\fi


\begin{document}
\sffamily   % for use with a résumé using sans serif fonts;
%\rmfamily  % for use with a résumé using serif fonts;
\hfill%
\begin{minipage}[t]{.6\textwidth}
\raggedleft%
{\bfseries Michael Yenik}\\[.35ex]
\small\itshape%
820 Lenora St Unit 2309\\
Seattle, WA 98121\\[.35ex]
\Mobilefone~(704)-564-3560\\
\Letter~{mgyenik77@gmail.com}
\end{minipage}\\[1em]
%
\begin{minipage}[t]{.4\textwidth}
\raggedright%
{\bfseries Glowforge}\\[.35ex]
%\small\itshape%
%street and number\\
%postcode city
\end{minipage}
\hfill % US style
%\\[1em] % UK style
\begin{minipage}[t]{.4\textwidth}
\raggedleft % US style
\today
%April 6, 2006 % US informal style
%05/04/2006 % UK formal style
\end{minipage}\\[2em]
\raggedright
Dear Sir or Madam:\\[1.5em]
%
When I was just a kid, I watched a show called BattleBots which sparked a
lifelong interest in engineering and garage tinkering. I started to become
interested in circuit design when I was 13, and soon afterwards discovered what
a microcontroller was. Even though I was still fairly young, it was a paradigm
shift for me - I could get an entire "computer" for just a few dollars that is
as small as a fingernail. Ever since then, embedded design has been my passion.
I've been designing circuits, etching boards, and programming microcontrollers
ever since high school.

When I got to college, I had my first exposure to a makerspace equipped with a
laser cutter, a 60W system from ULS. To this day it remains one of the most
useful tools I've used, and when I'm thinking about my dream garage, a laser
cutter is a non-negotiable part of it.

Now that I've been working full time for a while, I've begun to wonder if I'm
really spending my time in a way that I think is meaningful. Although the
problems that Google is solving are interesting from a technical perspective, I
feel that I've drifted away from the things that made me passionate about my
work in the first place. My hobbies have persisted into adulthood, and each time
I meet with fellow makers to work on projects, I am reminded of what that
passion feels like.

That's why I'm interested in this position at Glowforge - it's a product that I
personally care about, doing embedded work that I'm passionate about, which
engages with the maker community that I feel I am a part of.

I think the skills I have are a great match too - I have experience working in
the Linux kernel for about 2+ years at Google.  I have many years of embedded
experience, both professionally and personally. I've done a personal project
that used AWS IoT to control some window shades. I've built several 3D printers,
even modifying the firmware, making custom hardware enhancements, and dabbling
with my own discrete stepper motor controllers for fun. I wrote an operating
system with my friend for fun in college (it's ancient code, but can be dug up
on my github). I even have some experience with image processing and machine
learning.

I can only mention a handful of the things I've worked on in my resume, and I'm
not able to keep my github up to date because of Google's notoriously strict IP
policy for employees (Google generally asserts copyright, even on work I do in
my own time with my own resources), but I hope you'll see that my experience and
Glowforge's goals seem to greatly overlap. I would be very interested in the
opportunity to talk more about what I've worked on and ask more about what
Glowforge is working on. If you choose to reach out, I look forward to talking
more.

%Yours sincerely,\\[2em] % if the opening is "Dear Mr(s) Doe,"
Yours faithfully,\\[0em] % if the opening is "Dear Sir or Madam,"
%
%\includegraphics[scale=0.75]{signature_blue}\\
{\bfseries Michael Yenik}\\
%
\vfill%
\end{document}
